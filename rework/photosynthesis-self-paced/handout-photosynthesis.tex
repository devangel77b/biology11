Overview
You have several resources available to you for this unit. At a minimum, you need to complete the graded assignments on the Checklist. 
A vocabulary list for each textbook section is given below as well as detailed instructions for each assignment.
There are videos linked below that explain all of the material in the unit. They are not required, but are extremely helpful for you to understand the concepts. Additional online materials provide another vehicle for you to understand the material.
Introduction to the Unit
1. During the first class of this unit Dr. Hahn will be giving you a lecture overview of cellular energetics using these slides
2. It is strongly suggested that you read textbook section 4.1 to better understand these concepts as they are the basis for the unit regardless of the topic you are studying.
3. Redox can be a challenging topic, but it is important to understand for this unit. This video is helpful and highly recommended.
   1. The Penguin Prof. “Redox Relief.” YouTube. https://www.youtube.com/watch?v=eL0BH5O9Sdo (10min 37sec) Accessed 27 Mar 2021.


Part I: Overview of Photosynthesis
4. Photosynthesis Table
   1. Fill in this table with the basic information on photosynthesis as you work through the unit
   2. You will be allowed to use an approved version of this table on the test at the end of this unit
5. Textbook section 5.1
   1. Vocabulary: photosynthesis, autotroph, photoautotroph, heterotroph, mesophyll, stomata, chloroplast, thylakoids, chlorophyll, granum (plural = grana), stroma, Calvin cycle, light-dependent reactions
   2. Important concepts
      1. Think about how energy from the sun (whether directly or indirectly) powers all reactions on Earth.
      2. Memorize the formula for photosynthesis in both words and symbols - - though that may seem intimidating, break it up and think about what you already know: What do plants need? What do they make? Then balance the atoms.
      3. Look at the structures in the leaf diagram to understand the links between organisms, tissues, cells, organelles, and molecules.
   3. Note: the “Concepts in Action” link dives right into the details for the full set of reactions. You can use it if you would like, but it may work better as a review at the end of the unit. 
   4. You do not need to turn in your notes from this section. 
6. Photosynthesis POGIL Part I 
   1. You will be slowly working through a lengthy POGIL as you progress through the unit. Only the first part is linked above.
   2. You do not need to turn in your work on the POGIL, but you do need to submit your answers to selected prompts via this Google Form. Pay separate attention to the answers to these questions (which are provided after you submit your work) as they are the main points of the activity for this course.
7. Quiz Part I 
   1. When you are ready to take the quiz, email Dr. Hahn and she will give you access.
   2. Do not open this quiz until you are ready to take it.
   3. This quiz will be based on the textbook reading and Part I of the POGIL. 
   4. You can complete this quiz at any time during the unit, but it must be done before you take the final test on the unit. Do not discuss the content of the quiz with anyone.
Part II: Light-Dependent Reactions
8. Textbook section 5.2
   1. Vocabulary: 
   2. 9. Photosynthesis POGIL Part II 
   1. This is the continuation of the POGIL you began in Part I, so you will likely need to look back at your work 
   2. You do not need to turn in your work on the POGIL, but you do need to submit your answers to selected prompts via this Google Form
10. Quiz Part II 
   1. When you are ready to take the quiz, email Dr. Hahn and she will give you access.
   2. Do not open this quiz until you are ready to take it without notes.
   3. This quiz will be based on the textbook reading and Part II of the POGIL. 
   4. You can complete this quiz at any time during the unit, but it must be done before you take the final test on the unit. Do not discuss the content of the quiz with anyone.


Part III: Calvin Cycle
11. Textbook section 5.3
   1. Vocabulary: 
   2. 12. Photosynthesis POGIL Part III 
   1. This is the continuation of the POGIL you began in Parts I and II, so you will likely need to look back at your work 
   2. You do not need to turn in your work on the POGIL, but you do need to submit your answers to selected prompts via this Google Form
13. Quiz Part III 
   1. When you are ready to take the quiz, email Dr. Hahn and she will give you access.
   2. Do not open this quiz until you are ready to take it.
   3. This quiz will be based on the textbook reading and Part III of the POGIL. 
   4. You can complete this quiz at any time during the unit, but it must be done before you take the final test on the unit. Do not discuss the content of the quiz with anyone.


Part IV: Putting it all together
14. Photosynthesis Case Study
   1. You can work together on the case study (linked above), but each student must submit their own document with answers in their own words and a graph. Upload a Google Doc to the posting on Classroom. This will be graded.
   2. A background video specifically on beets and nonspecifically on photosynthesis: National Center for Case Study Teaching in Science, University of Buffalo. “Sweet Beets.” YouTube. https://www.youtube.com/watch?v=BFn9jEhDJmI Accessed 27 Mar 2021.
15. Photosynthesis gif Project
   1. For the final project of this unit you will construct a gif of either the light-dependent reactions or the Calvin cycle. You can find all of the instructions and the grading criteria on the document linked above.
   2. All students, whether studying photosynthesis or cellular respiration, will be creating a gif. You can choose to work with classmates who are doing the other part of photosynthesis and/or consult with students who are doing the same part as you are, but your final work should be original.


Part V: Comparison of Photosynthesis and Cellular Respiration
16. You will need to critique the gifs created by your classmates following these instructions. You will be randomly assigned three gifs to critique based on the same criteria on which your gif will be scored (included in the instructions linked above). Your critiques will be graded based on accuracy and how constructive they are. Include the identifying information for the gif you are critiquing, your specific comments and a score for the 1) Script, 2) Process Accuracy, and 3) Quality of the gif. (5pts each, 15 points total) 


Videos:
You do not need to watch all the videos, but you are strongly encouraged to watch several of them. You will likely need to watch them several times. The main sources I use for videos (Amoeba Sisters, Bozeman Science, Crash Course, and Khan Academy), are all helpful and cover essentially the same information, but you may prefer the style of one over the other. Remember the easier is not always better. These videos are not specific for any Part (with the exception of “Autotrophs and Heterotrophs” Amoeba Sisters video, which corresponds to the material in Part I), so if you do not understand all of the material initially, that is fine. By the end of the unit the information in the videos will be review. When watching these videos, take notes!!!
1. Khan Academy (If you go to Khan Academy directly there are more videos, articles, practice, and quizzes. These resources are highly recommended.)
   1. “Photosynthesis.” YouTube. https://www.youtube.com/watch?v=-rsYk4eCKnA&list=PLAE36CEFE9200FDDD (13min 37sec) Accessed 28 Mar 2021.
   2. You will need to watch more than just this video to understand the details of this process
2. Bozeman Science. “Photosynthesis” YouTube https://www.youtube.com/watch?v=g78utcLQrJ4 (12min 26sec) Accessed 27 Mar 2021.
      1. Note: photorespiration goes beyond the Standard level for this course, but could be used by Earned Honors students
3. Crash Course. “Photosynthesis: Crash Course Biology #8.” YouTube. https://www.youtube.com/watch?v=sQK3Yr4Sc_k (13min 14sec) Accessed 27 Mar 2021.
4. The Penguin Prof. “Photosynthesis: Fun in the Sun.” YouTube. https://www.youtube.com/watch?v=FfLLHQDgpjI (14min 36sec) Accessed 27 Mar 2021.
5. Amoeba Sisters 
   1. “Autotrophs and Heterotrophs” YouTube. https://www.youtube.com/watch?v=f8G7IulYxiA (6min 22sec) Accessed 23 Mar 2021.
      1. This is a general information video to put types of organisms into perspective
   2. “Photosynthesis and the Teeny Tiny Pigment Pancakes” YouTube. https://www.youtube.com/watch?v=f8G7IulYxiA (7min 45sec) Accessed 23 Mar 2021.
      1. Light on details


Additional videos and resources
1. Graham Johnson. “ATP Synthase.” YouTube. https://www.youtube.com/watch?v=CN2XOe_c0iM Accessed 25 Mar 2021.
2. Omar Ali. “Electron Transport Chain.” YouTube. https://www.youtube.com/watch?v=rdF3mnyS1p0&t=112s Accessed 25 Mar 2021.
* Though this is focused on cellular respiration, the electron transport chain in photosynthesis is very similar