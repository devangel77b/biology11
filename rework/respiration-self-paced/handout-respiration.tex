Overview
You have several resources available to you for this unit. At a minimum, you need to complete the assignments on the Checklist. 
A vocabulary list and some concepts to make a special note of for each textbook section is given below as well as detailed instructions for each assignment in each of the Parts.
There are videos linked below that explain all of the material in the unit. They are not required, but are extremely helpful for you to understand the concepts. Additional online materials listed below can also provide another vehicle for you to understand the material.
Introduction to the Unit
1. During the first class of this unit Dr. Hahn will be giving you a lecture overview of cellular energetics using these slides
2. It is strongly suggested that you read textbook section 4.1 to better understand these concepts as they are the basis for the unit regardless of the topic you are studying.
3. Redox can be a challenging topic, but it is important to understand for this unit. This video is helpful and highly recommended.
   1. The Penguin Prof. “Redox Relief.” YouTube. https://www.youtube.com/watch?v=eL0BH5O9Sdo (10min 37sec) Accessed 27 Mar 2021.


Part I: Overview of Cellular Respiration and Glycolysis
1. Cellular Respiration Table 
   1. Fill in this table with the basic information on cellular respiration as you work through the unit
   2. You will be allowed to use an approved version of this table on the test at the end of this unit
2. Textbook section 4.2
   1. Vocabulary: exergonic, activation energy, endergonic, metabolism, ATP, free energy, AMP, ADP, hydrolysis, glycolysis, glucose, pyruvate, catabolize, NAD+, NADH, oxidation, reduction
   2. Important concepts
      1. Why must energy be stored as opposed to being available as free energy?
      2. What is the implication from the fact that nearly all eukaryotic cells and nearly all prokaryotic cells perform glycolysis?
      3. Make sure to log the reactants and products and locations for glycolysis on your table.
   3. Note: This section relies on an understanding of the previous section, which was in the introductory lecture to this unit. You may need to go back to that textbook section to review some of the concepts if they do not make sense when used in this and subsequent readings.
   4. You do not need to submit your notes on this reading assignment
3. POGIL Part I 
   1. You will be working through POGILs as you progress through the unit. This particular POGIL will give an overview of all of the stages of cellular respiration (including those covered in the reading in Part II), but only ATP and glycolysis will be covered on the first quiz.
   2. You do not need to turn in your work on the POGIL, but you do need to submit your answers to selected prompts via this Google Form
4. Quiz Part I 
   1. When you are ready to take the quiz, email Dr. Hahn and she will give you access.
   3. Do not open this quiz until you are ready to take it.
   4. This quiz will be based on the textbook reading and Part I of the POGIL. 
   5. You can complete this quiz at any time during the unit, but it must be done before you take the final test on the unit. Do not discuss the content of the quiz with anyone.
Part II: Citric Acid Cycle and Oxidative Phosphorylation
5. Textbook section 4.3
   1. Vocabulary: mitochondria, aerobic, anaerobic, coenzyme A, acetyl CoA, citric acid cycle (aka Kreb cycle), mitochondrial matrix, FAD, FADH2, anabolic, catabolic, electron acceptor, inner mitochondrial membrane, electrochemical gradient, oxidative phosphorylation, electron transport chain, intermembrane space, hydrogen ions, ATP synthase, chemiosmosis 
   2. Important concepts
      1. Though oxygen is not strictly required by all of the reactions discussed here, if the reactions cannot proceed because their reactants need to be recycled by reactions that require oxygen, they are considered to be aerobic
      2. Make sure to log the reactants and products and locations for the citric acid cycle and oxidative phosphorylation on your table.
      3. Pay special attention to why we need oxygen in order to survive… the reason may surprise you.
      4. The ideal number of ATP produced from one glucose molecule is 38, but the actual number can vary. Reasons for this are discussed in the section “ATP Yield”.
6. Cellular Respiration POGIL Part II 
   1. This is the continuation of the POGIL you began in Part I, so you will likely need to look back at your work 
   2. You do not need to turn in your work on the POGIL, but you do need to submit your answers to selected prompts via this Google Form
7. Quiz Part II 
   1. When you are ready to take the quiz, email Dr. Hahn and she will give you access.
   3. Do not open this quiz until you are ready to take it.
   4. This quiz will be based on the textbook reading and Part II of the POGIL. 
   5. You can complete this quiz at any time during the unit, but it must be done before you take the final test on the unit. Do not discuss the content of the quiz with anyone.


Part III: Fermentation
8. Textbook section 4.4
   1. Vocabulary: fermentation, lactic acid fermentation (reactants, products), alcohol fermentation (reactants, products), anaerobic cellular respiration, facultatively anaerobic, obligate anaerobe 
   2. Important concepts:
      1. Think about why these processes need to occur (hint: it relates to NADH and glycolysis), and why different processes occur in different types of cells
      2. Are the products like lactic acid and alcohol the main reason these processes occur?
      3. There isn’t a specific place for these types of reactions on the Cellular Respiration Table, but you can use the second page to make notes about them.
9. Cellular Respiration POGIL Part III 
   1. This is the continuation of the POGIL you began in Part I, so you will likely need to look back at your work 
   2. You do not need to turn in your work on the POGIL, but you do need to submit your answers to selected prompts via this Google Form
10. Quiz Part III 
   1. When you are ready to take the quiz, email Dr. Hahn and she will give you access.
   2. Do not open this quiz until you are ready to take it.
   3. This quiz will be based on the textbook reading and Part III of the POGIL. 
   4. You can complete this quiz at any time during the unit, but it must be done before you take the final test on the unit. Do not discuss the content of the quiz with anyone.


Part IV: Putting it all together
11. Cellular Respiration Case Study
   1. You can work together on the case study (linked above), but each student must submit their own document with answers in their own words. Upload a Google Doc to the posting on Classroom. This will be graded on correctness. (20pts)
12. Cellular Respiration gif Project
   1. For the final project of this unit you will construct a gif of either glycolysis, the Krebs cycle, or oxidative phosphorylation. You can find all of the instructions and the grading criteria on the document linked above. (30pts)
   2. All students, whether studying photosynthesis or cellular respiration, will be creating a gif. You can choose to work with classmates who are doing one of the other parts of cellular respiration and/or consult with students who are doing the same part as you are, but your final work should be original.




Part V: Comparison of Photosynthesis and Cellular Respiration
13. You will need to critique the gifs created by your classmates following these instructions. You will be randomly assigned three gifs to critique based on the same criteria on which your gif will be scored (included in the instructions linked above). Your critiques will be graded based on accuracy and how constructive they are. Include the identifying information for the gif you are critiquing, your specific comments and a score for the 1) Script, 2) Process Accuracy, and 3) Quality of the gif. (5pts each, 15 points total)


Videos and Additional Resources:
You do not need to watch all the videos, but you are strongly encouraged to watch several of them. You will likely need to watch them several times. The main sources I use for videos (Amoeba Sisters, Bozeman Science, Crash Course, Khan Academy, and The Penguin Prof), are all helpful and cover essentially the same information, but you may prefer the style of one over the other. Remember the easier is not always better. These videos are not specific for any Part, so if you do not understand all of the material initially, that is fine. By the end of the unit the information in the videos will be review. When watching these videos, take notes!!!
1. Khan Academy (If you go to Khan Academy directly there are more videos, articles, practice, and quizzes. These resources are highly recommended.)
   1. “Introduction to cellular respiration, Cellular respiration, Biology, Khan Academy.” YouTube. https://www.youtube.com/watch?v=2f7YwCtHcgk (14min 18sec) Accessed 27 Mar 2021
   2. You will need to watch more than just this video to understand the details of this process
2. Bozeman Science. “Cellular Respiration.” YouTube. https://www.youtube.com/watch?v=Gh2P5CmCC0M&t=255s (14min 13sec) Accessed 27 Mar 2021.
   1. Comparison of runner efficiency adds context
3. Crash Course. “ATP & Respiration: Crash Course Biology #7.” YouTube. https://www.youtube.com/watch?v=00jbG_cfGuQ (13min 25sec) Accessed 27 Mar 2021.
4. The Penguin Prof. “Cellular Respiration: What Food is For”. YouTube. https://www.youtube.com/watch?v=aA8d-tt6dII (16min 9sec) Accessed 28 Mar 2021.
   1. Gives an overview of autotrophs and heterotrophs
   2. Includes a historical background of a scientist
5. Amoeba Sisters. “Cellular Respiration and the Mighty Mitochondria.” YouTube. https://www.youtube.com/watch?v=4Eo7JtRA7lg  Accessed 28 Mar 2021.
   1. The ETC is very oversimplified in this video


Additional videos and resources
1. Graham Johnson. “ATP Synthase.” YouTube. https://www.youtube.com/watch?v=CN2XOe_c0iM Accessed 25 Mar 2021.
2. Omar Ali. “Electron Transport Chain.” YouTube. https://www.youtube.com/watch?v=rdF3mnyS1p0&t=112s Accessed 25 Mar 2021.
3. “Autotrophs and Heterotrophs” YouTube. https://www.youtube.com/watch?v=f8G7IulYxiA (6min 22sec) 
   * This is a general information video to put types of organisms into perspective