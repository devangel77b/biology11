\documentclass[quiz,addpoints,noanswers]{exam}

\usepackage[quiz]{../biology11}
\usepackage{chemformula}

\title{Quiz}
\date{\today}
\author{\mobeardInstructorShort}

\begin{document}
\maketitle
\begin{abstract}
To keep you guys on your toes. This quiz should only take about \SI{5}{\min}
\end{abstract}

\begin{questions}
\question[1] Which of the following statements regarding evolution is \textbf{FALSE}?
\begin{choices}
\CorrectChoice Evolution has a ``drive'' and always acts to improve performance.
\choice Evolution is descent with modification.
\choice Evolution is a consequence of having heritable traits and differential survival of offspring.
\choice Evolution occurs over geologic timescales, but can also occur rapidly such as in COVID-19.
\end{choices}

\question[1] Based on \emph{scientific evidence}, approximately how old is life on earth?
\begin{choices}
\CorrectChoice Prokaryotic life is around 3.9 billion years old, multicellular organisms have only been around for about 500 million years.
\choice 5000 years.
\choice 17 years.
\choice Everything is 1 day old and was created yesterday by an evil genius out to trick us. 
\end{choices}

\question[1] Which of the following is \textbf{NOT} a way that single cells can bend the limits of diffusion?
\begin{choices}
\choice Being large in one dimension but skinny in others.
\choice Using ATP to power transmembrane proteins to actively pump materials into a cell.
\choice Using special organelles or increasing flow to improve intake / output of materials. 
\CorrectChoice By growing very large, cells can avoid the need to use diffusion. 
\end{choices}

\question[1] Which of the following about photosynthesis is \textbf{FALSE}?
\begin{choices}
\CorrectChoice Light reactions do not require light to proceed, they are named for Calvin Light, FRS, who was awarded the Nobel Prize in 1953 for his work on photosynthesis.
\choice CAM and C4 processes are alternatives to the Calvin cycle that allow \ch{CO2} uptake in the dark, such as at night. 
\choice Overall, photosynthesis uses energy from photons to synthesize \ch{CO2} and water into glucose, liberating oxygen. 
\choice In eukaryotic photoautotrophs, the chloroplast is the organelle which is the site of photosynthesis and is believed to have arisen in an endosymbiotic event with a blue-green alga. 
\end{choices}

\question[1] Which is likely to be most severe, and why?
\begin{choices}
\choice Drowning in seawater, because isotonic solutions can cause cells to burst
\CorrectChoice Drowning in fresh water, because hypotonic solutions can cause cells to burst
\choice Drowning in fresh water, because hypotonic solutions can cause cells to shrink or dry out
\choice Drowning in fresh water, because  hypertonic solutions can cause cells to shrink or dry out
\end{choices}


\end{questions}
\end{document}