\documentclass[quiz,addpoints,noanswers]{exam}

\usepackage[quiz]{../biology11}
\usepackage{chemformula}

\title{Quiz (earned honors)}
\date{\today}
\author{\mobeardInstructorShort}

\begin{document}
\maketitle
\begin{abstract}
To keep you guys on your toes. This quiz should only take about \SI{5}{\min}
\end{abstract}

\begin{questions}
\question[5] We discussed photosynthesis (which uses light energy and produces glucose and oxygen) and cellular respiration (which consumes sugars).  Can life survive without oxygen? Can life survive without sunlight? Support your hypothesis with examples from the history of life on earth, biochemistry examples, or example organisms. 
\begin{solution}[6in]
Yes life can survive without oxygen. As examples, in the history of life on earth, prokaryotes developed and proliferated well before there was free atmospheric oxygen (which came in around 2 billion years ago due to the appearance of photosynthesis). Biochemistry examples include fermentation, in which cellular respiration is able to carry out glycolysis etc but does not perform \ch{O2} requiring steps, instead using alternative reactions that produce lactic acid or ethanol; these are economically important in the production of vinegar, sauerkraut, pickles, wine and beer for example... example extant organisms include the microorganisms that make these products (e.g. \emph{Leuconostoc}, \emph{Saccharomyces}, many others, etc...). Life can also survive without sunlight. Though most life on earth depends on it, there are examples of deep sea hydrothermal vent fauna that are completely independent of the sun, instead using energy from compounds emitted by the vents. 
\end{solution}
\end{questions}
\end{document}