\documentclass[quiz,addpoints,noanswers]{exam}

\usepackage[quiz]{../biology11}
\usepackage{chemformula}

\title{Quiz: Blast from the past! (earned honors)}
\date{\today}
\author{\mobeardInstructorShort}

\begin{document}
\maketitle
%\begin{abstract}
%Today's quiz is a salute to Q1/Q2. This quiz should only take about \SI{5}{\minute}. Answers will be made available on Google Drive \SI{15}{\minute} after the quiz start time. 
%\end{abstract}

\begin{questions}
\question[1] Which of the following statements regarding translation and transcription is \textbf{FALSE}?
\begin{choices}
\CorrectChoice There is no difference between translation and transcription.
\choice Transcription is the process by which mRNA is generated from DNA.
\choice Translation is the process by which proteins are assembled based on mRNA.
\choice There is a one-to-one correspondence between DNA bases and RNA bases, but there is NOT a one-to-one correspondence between RNA bases and amino acids. .
\end{choices}

\question[1] Macromolecules: which of the following about cellulose is \textbf{FALSE}?
\begin{choices}
\CorrectChoice Cellulose consists of subcutaneous adipose (fat) tissue.
\choice The chemical formula of celluose is \ch{(C6H10O5)n}. 
\choice Cellulose is a major structural material in bacteria and plants, and providing stiffness in the form of fibers with aligned covalent bonds that can resist mechanical forces.
\choice Cellulose is obtained by dehydration synthesis using glucose monomers produced through photosynthesis. 
\end{choices}

\question[1] How can DNA be used to test hypotheses about evolution? Please select all that apply. 
\begin{choices}
\choice DNA is not useful in testing hypotheses about evolution, because lateral gene transfer happens a bazillion times a second. 
\CorrectChoice Comparing two DNA sequences can provide an estimate of how long ago they had a common ancestor.
\CorrectChoice Shared, derived changes can be identified in DNA sequences, just like in any other trait, and used to construct phylogenetic trees. 
\CorrectChoice Overall data about DNA like chromosomes and evidence of genome duplication, etc can also be used to build support or refute hypotheses about relationships among clades. 
\end{choices}

\question[1]  Mathematically, a codon of three DNA base pairs could encode up to $4\cdot4\cdot4=64$ different things, but only about 20 amino acids are used. Is DNA wasteful as an information encoding scheme, or are there any potential benefits to this encoding scheme?
\begin{solution}[1in]
The extra-ish base pair provides important redundancy to help with ``error correction / avoid translation going completely out of whack due to point mutations. For example, for some of the amino acids encoded (based on the wheel diagram) a single point mutation would not change the amino acid. In other cases, a single point mutation of the right sort can switch what is encoded, or cause a stop or start to be encoded instead. 
\end{solution}

\question[1] In order to function, nerve cells become polarized to about \SI{-70}{\milli\volt} by filling up with \ch{K+} and \ch{Na+} ions. How is this resting potential accomplished?
\begin{choices}
\choice Simple diffusion through the cell membrane
\CorrectChoice Active transport, to pump the ions against a gradient, powered by ATP 
\choice Facilitated diffusion, by opening ion channels, the ions can flow into the cell against the gradient
\choice Vesicles form and the ions are engulfed via phagocytosis.
\end{choices}
\end{questions}
\end{document}