\documentclass[quiz,addpoints,noanswers]{exam}

\usepackage[quiz]{../biology11}
\usepackage{chemformula}

\title{Quiz: Blast from the past!}
\date{\today}
\author{\mobeardInstructorShort}

\begin{document}
\maketitle

\begin{questions}
\question[4] In the space below, summarize what you remember about DNA translation and transcription. You may use sketches if you like. 
\begin{solution}[4in]
DNA translation is the process by which RNA is generated from DNA. The complementary RNA base pair for each DNA base pair is used (A-U, T-A, C-G, G-C).

Transcription is when the RNA is used to assemble proteins. Transfer RNA is used to connect with each 3-base codon in series, in order to assemble the protein using each corresponding amino acid, one at a time. 
\end{solution}

\question[1] How can DNA be used to test hypotheses about evolution? Please select all that apply. 
\begin{choices}
\choice DNA is not useful in testing hypotheses about evolution, because lateral gene transfer happens a bazillion times a second. 
\CorrectChoice Comparing two DNA sequences can provide an estimate of how long ago they had a common ancestor.
\CorrectChoice Shared, derived changes can be identified in DNA sequences, just like in any other trait, and used to construct phylogenetic trees. 
\CorrectChoice Overall data about DNA like chromosomes and evidence of genome duplication, etc can also be used to build support or refute hypotheses about relationships among clades. 
\end{choices}

\end{questions}
\end{document}