Biology
2020-2021
Mrs. Popielski
COURSE OUTLINE.
Throughout this course, students will gain a scientific understanding of life and the laws,
theories and principles that govern it. This will be accomplished by exploring concepts in Biology
and examining the history and nature of science. Concepts are presented in a conceptual
approach, showing how the topics are connected to each other and allowing students to build
on what they already know. Students will learn the major and guiding principles of molecular
and cellular biology and how they work together to control the functions of all living things.
Evolution is considered one of the guiding principles and will be incorporated into the content
throughout the year, aiding students in making connections between the various concepts. The
content discussed in this course includes:
•
•
•
•
•
•
•
•
Uncertainty and Nature of Science
Characteristics of Life
Chemistry of Life
Molecular Genetics
Mendelian Genetics
Cellular Structure and Function
Cellular Energy
Ecology
CLASS MATERIALS.
The following materials are required for class. Students are encouraged to take their notes
using their iPad or laptop, but may use a notebook or binder if he/she chooses.
•
•
iPad/Laptop.
o Required Apps:
▪ Google Classroom, Google Calendar, Google Sites and Google Drive
▪ Notability
▪ iWork Suite or Microsoft Office
▪ Zoom
▪ Go Formative
Textbook: Concepts of Biology
o Free Online version can be found here
o Print ISBN 1938168119
GRADING BREAKDOWN.
Tests .........................................................................20%
Quizzes ......................................................................20%
Projects and Essays......................................................20%
Laboratory Participation & Reports....................................20%
Homework...................................................................10%Engagement, Growth & Curiosity......................................10%
CLASS MEETINGS.
The course will meet either 2 or 3 times a week, depending on the A or B week schedule.
If you are attending class virtually, classes can be accessed using the following Zoom link:
https://zoom.us.my/popielski
CLASS WEBSITES, USES AND EXPECTATIONS.
There are three major Google applications that are associated with this, and all courses at MBS.
• Both sections of Biology (periods 5 and 8) will use the same Google Site. The link to the
Google Site can be found below:
•
•
o https://sites.google.com/mbs.net/popielski/home
o The Google Site will be used throughout the year and is a good spot to locate
information and additional resources for this course.
o The site also includes Mrs. Popielski’s teaching schedule and availability.
Each section of the class has its own Google Classroom and associated Google Calendar.
All assignments will be posted on Google Calendar and be available through Google
Classroom.
EARNED HONORS IN BIOLOGY.
Throughout the course, students will have the option earn honors in Biology. General
expectations for both the Standard and Honors students are included on the Google Classroom
and Google Site for this course. Additionally, you can access the information here. For each unit
and/or assessment, differentiated opportunities will be explained in more detail through the
learning goals of specific units.
HOMEWORK AND ASSIGNMENT POLICY.
•
•
•
•
•
•
•
•
•
All assignments for this class will be available on Google Classroom along with any
important information and due dates.
All due dates for assignments will be posted on Google Calendar, linked with Google
Classroom.
All homework is due the class after it was assigned, unless otherwise stated by the teacher
and Google Calendar.
Missing homework will result in zero points earned until the assignment is turned in by the
student.
Late homework, labs or projects will be accepted but will result in a 10% credit deduction for
every school day that it is late.
If an assignment is turned in late due to an excused absence, it will not receive point
deductions.
Excused absences include pre-scheduled college visits and any other days considered to be
excused as per the school policy.
In case of an absence, it is the student’s responsibility to obtain any notes, work and
assignments.
Assignments missed because of an absence need to be made up as soon as possible. If aquiz or test was missed due to an absence, students should schedule a time with the
instructor to make-up the assessment either: before/after school, during a free period or
during collaborative period.
• If a student is struggling with material that was covered while he/she was out of class, they
can receive extra help during collaborative period or by scheduling a time to meet
before/after school or during a free period.
QUIZZES AND TESTS.
•
•
•
•
•
•
Go Formative, an online testing platform used by many teachers at MBS, will be used to
administer tests and quizzes.
Prior to each test, students will be given a study guide as well as sample questions to help
them prepare.
For each quiz, students will have the opportunity to complete an optional correction sheet to
earn back 1/2 credit on any questions marked incorrect. Quiz corrections should be used as
a learning experience to help students prepare for upcoming tests.
On tests, students can complete corrections for all scores below an 80%.
Test/Quiz corrections must be submitted along with the original graded assessment.
Students will have 1 week after receiving their graded assessment to successfully complete
any corrections.
CLASSROOM RULES AND POLICIES.
Below are a general list of classroom rules and procedures that she be followed at all times.
Students will receive more specific guidelines, and a separate contract, on Laboratory Rules
and Procedures.
• Students should behave and dress according to the school guidelines outlined in the MBS
Student and Parent Handbook.
• Arrive to class on time, prepared with all materials, assignments or projects.
• Behavior should promote learning and inquire and not disrupt it.
• All behavior and discussion in class is should be respectful of all members of the learning
community.
• On lab days, students are expected to dress in accordance to the safety guidelines that will
be discussed in class (ie: no open-toe shoes).
• Laboratory rules and expectations must be followed at all times during lab.
• In class, all technology should be used for the advancement the learning process (taking notes,
virtual labs, educational resources, etc.)
• No form of academic dishonesty or plagiarism will be tolerated. Because there is often a lot of
collaboration and group work in science, students should turn in their own assignments, in
their own words, unless told otherwise by the instructor.
Failure to follow any of the classroom rules will result in a verbal warning the first time a rule is
broken.
Failure to correct the problem will result a parent/guardian contact through the school’s
electronic PCF system. Any severe disruptions will immediately be brought to the attention of
both school administration and guardians. Severe disruptions are considered to be: overt refusal
to follow instructions, fighting, vandalism and any behavior that threatens/disrupts the safety ofstudents, the learning environment and the school.
TEACHER CONTACT INFORMATION.
Ms. Popiekski
Science Department Faculty
MSC 106
E-mail: apopielski@mbs.net
School Phone: (973) 539 - 3032 ext. 864
