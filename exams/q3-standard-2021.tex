\documentclass[exam,addpoints,noanswers]{exam}

\usepackage[exam]{biology11}
\usepackage{chemformula}

\title{Test \#3}
\date{\today}
\author{\mobeardInstructorShort}

\begin{document}
\maketitle
\vfill
\mobeardExamNameBlock
\vfill
Instructions: 
\begin{enumerate}
\item Do not open exam until instructor announces that you may begin.
\item Closed notes, closed book.  You may use two \SI{8.5x11}{\inch} note sheets, both sides.
\item Multiple choice are worth one point, short answer questions are worth substantially more. I recommend you quickly read through the multiple choice questions and use the information contained within them to aid you in answering the essay questions, but do not spend excessive time answering the multiple choice. If time permits, please answer all before turning in your test to maximize your score.
\item In answering the short answer questions, be thorough but concise. Deep understanding of the concepts will be displayed by proper use of vocabulary and discussion of the interconnectedness of concepts.
%\item Please write your name on each page in case the sheets are separated. 
\end{enumerate}
\vfill
\begin{center}
\gradetable[h][questions]
\end{center}
\clearpage







\begin{questions}
\question[1] Refer to the phylogenetic tree of life above. Which statement is consistent with the tree? 
\begin{choices}
\CorrectChoice The most recent common ancestor of chordates and echinoderms had deuterostomic development (second hole becomes mouth) as a shared, derived trait.
\choice The most recent common ancestor of chordates and echinoderms shed their skin as a shared, derived trait.
\choice The most recent common ancestor of chordates and echinoderms shared radial symmetry and neuromuscular cells as a shared, derived trait. 
\choice Chordates and echinoderms do not share a common ancestor. 
\end{choices}


\question[1] Which of the following statements regarding evolution is FALSE?
\begin{choices}
\choice Evolution does not have a ``drive'' and does not always act to improve one particular function
\choice Evolution is descent with modification
\CorrectChoice Evolution does not require having heritable traits nor differential survival of offspring
\choice Evolution can occur rapidly or over geologic timescales
\end{choices}



\question[1] Which of the following are examples of scientific evidence that life on earth has been around for a long time (3.9 billion years in case of prokaryotes, 550 million years in case of multicellular life). Please mark all that apply.
\begin{choices} 
\CorrectChoice Fossil bacteria in rocks that are 2-4 billion years old
\CorrectChoice Sea creatures in rocks that have been raised to the top of the Andes due to geologic processes
\CorrectChoice Molecular divergence times for animals of about 600 million years before present; and for prokaryotes of 2-4 billion years before present
\choice Written genealogies (family histories for historical or mythical people) stretching back 4000-5000 years before present. 
\end{choices}




\question[1] Which of the following is true about the molecule best described as the ``energy currency'' of the cell? Please mark all that apply. 
\begin{choices}
\CorrectChoice Adenosine triphosphate can be broken into adenosine diphosphate and a phosphate group, releasing energy
\choice The enzyme adenosinase removes adenosine from adenosine diphosphate and grafts it onto adenosine triphosphate, storing energy for later release
\CorrectChoice In adenosine triphosphate, the three phosphate groups are able to store energy because they are like unstable, boisterous 8th graders forced to sit on the same, too small bus bench seat; when a phosphate is removed, energy is thus released
\CorrectChoice Adenosine triphosphate can be synthesized from adenosine diphosphate by means of ATP synthase, an enzyme resident in the membranes of mitochondria and chloroplasts that is powered by a proton gradient
\end{choices}



\question[1] Which of the following about classification and taxonomy are true? Please mark all that apply.
\begin{choices}
\CorrectChoice Example phyla include porifera (sponges), cnidaria (anemones, jellyfish), mollusks, arthropods, and chordates 
\choice Species and genus are typically written in that order, as in \emph{ergaster Homo}, \emph{mola Mola}, \emph{gorilla Gorilla}, \emph{pelagica Chaetura}
\CorrectChoice While taxonomic ranks are well entrenched in biology textbooks and standardized tests, a criticism of them is that the ranks are not necessarily equivalent among groups, e.g. a ``class'' among vertebrates is not necessarily the same tree level as in insects or worms
\CorrectChoice Valid named groups (such as mammals and birds) that include all the living relatives of a single common ancestory are referred to as monophyletic and crown groups; other groups that leave things out (such as ``reptiles'') are called paraphyletic and are placed in quotes. 
\end{choices}



\clearpage
\question[1] Which of the following is NOT consistent with the flow of energy in an ecosystem?
\begin{choices}
\CorrectChoice Because cellular energetic processes are 100\% efficient, we expect to see equal biomass of primary producers, grazers, and predators
\choice In terrestrial ecosystems, we expect to see the highest biomass of primary producers, then grazers, then the least biomass will be predators. 
\choice Cellular energetic processes can shape global climate and drastically affect the abundance and distribution of life at the ecosystem level
\choice While most energy comes from the sun, there are ecosystems on earth that utilize alternative sources of energy
\end{choices}





\question[1] Which of the following are ways a single cell can bend the limits of diffusion? Please mark all that apply. 
\begin{choices}
\CorrectChoice Being large in one dimension but being skinny in other dimensions
\CorrectChoice Using ATP to power membrane spanning proteins to actively pump materials against concentration gradients
\CorrectChoice Using special organelles or increasing flow to improve intake or output of materials
\choice By growing very large, cells can avoid the need to rely on diffusion
\end{choices}




\question[1] Which of the following about cellular respiration is FALSE? 
\begin{choices}
\choice The Citric Acid Cycle is also named the Krebs Cycle, after Hans Krebs, who was eventually awarded the Nobel Prize for his work. 
\choice Glycolysis does not require oxygen to proceed. 
\CorrectChoice Overall, cellular respiration uses energy from photons to synthesize \ch{CO2} and water into glucose, liberating oxygen. 
\choice The mitochondrion is the site of cellular respiration, and includes inner membranes that provide the site for the enzymes used in the process as well as allowing for a proton gradient to power ATP production. 
\end{choices}




\question[1] When we made gravlax, we discussed salmon as an example of an anadromous fish that moves between freshwater and saltwater (and back). Why is this physiologically challenging? 
\begin{choices}
\choice In seawater, the water is hypotonic to the fish and causes it to intake excess water
\choice In seawater, the water is hypertonic to the fish and causes it to intake excess water
\CorrectChoice In freshwater, the water is hypotonic to the fish and causes it to intake excess water
\choice In freshwater, the water is hypertonic to the fish and causes it to intake excess water
\end{choices}




\question[1] Which of the following about the history of animal life on Earth is true? 
\begin{choices}
\choice Once an animal group comes ashore, they never return to the water
\choice Unrelated animal groups never evolve similar looking features
\choice Once a complex feature is evolved, it is never lost
\CorrectChoice Animal life appears to have appeared and diversified into all of the known phyla within a relatively short time
\end{choices}



\clearpage
\question[22] A Rube Goldberg machine is an overly complicated contraption that accomplishes a simple task using repurposed stuff. Select either photosynthesis or respiration (pick ONE), and explain whether you think an analogy to a Rube Golderberg machine applies. Is your chosen process ``well designed''? Hypothesize if the steps could have been borrowed from, or are similar to, other processes, and what you think this means. 

\clearpage
\question[23] When making gravlax, a piece of salmon filet (the main swim muscle of the fish) is placed in a bag with salt and sugar, and sometimes dill, black pepper, and a little bit of salt. This week, instead of making gravlax you decide to make kimchi (Korean fermented cabbage). As the first step, the cabbage is placed in a salty solution. As the last step, the cabbage is placed in a sealed, airtight container that lets in no oxygen. The final product smells sour almost like vinegar or sauerkraut. Explain the first step and the last step, using what you know about biology. 

\clearpage
\question[22] A little bit of energy is emitted by the sun. It falls to earth, is captured by a banana tree, turned into a banana, that is then eaten by an MBS student who then wins a hockey game. Tell a story about this bit of energy and how it is incorporated into, and transforms among, biomolecules to power the game winning goal shot. 

\clearpage
\question[23] Pick the most noteworthy thing you've heard about an animal or plant this term, explain what you know about the biology, and explain why you think it is interesting. 
\end{questions}
\end{document}
