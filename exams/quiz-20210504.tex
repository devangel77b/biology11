\documentclass[hw,addpoints,noanswers]{exam}

\usepackage[hw]{biology11}

\title{Quiz: Mitosis and meiosis}
\author{\mobeardInstructorShort}
\date{\today}
\duedate{in class}

\begin{document}
\maketitle

\begin{questions}
\question[1] What is mitosis? 
\begin{choices}
\choice The ten digits on the ends of my posterior appendages. 
\choice A process by which a diploid nucleus splits into four non-identical haploid nuclei.
\CorrectChoice A process by which a diploid nucleus splits into two identical diploid nuclei. 
\choice A disease caused by malnutrition originally described in engineering students in Cambridge, MA
\choice What's in a name? That which we call a rose by any other name would still smell as sweet. 
\end{choices}

\question[1] What are the phases of mitosis in the correct order?
\begin{choices}
\choice anaphase, metaphase, prophase, telophase
\choice prophase, metaphase, telophase, anaphase
\CorrectChoice prophase, metaphase, anaphase, telophase
\choice metaphase, prophase, anaphase, telophase
\end{choices}

\question[1] Which of the following are correct? Select all that apply.
\begin{choices}
\CorrectChoice During prophase, the chromosomes condense. 
\CorrectChoice During metaphase, the chromosomes line up along the equator.
\CorrectChoice During anaphase, the nuclear membrane dissolves.
\CorrectChoice During anaphase, the chromosomes are split and pulled along the spindle towards opposite poles.
\CorrectChoice During telophase, the nuclear membrane re-forms, resulting in two daughter nuclei.
\end{choices}

\question[1] What is meiosis? 
\begin{choices}
\choice An older, female sibling. 
\CorrectChoice A process by which a diploid nucleus splits into four non-identical haploid nuclei.
\choice A process by which a diploid nucleus splits into two identical haploid nuclei. 
\choice A disease caused by malnutrition from eating only mayonnaise. 
\choice What's in a name? That which we call a rose by any other name would still smell as sweet. 
\end{choices}

\question[1] Please explain two ways that meiosis generates genetic variation compared to asexual reproduction (e.g. by mitosis only). 
\begin{solution}
(1) Independent assortment, shuffling what alleles are present; and (2) recombination during crossover. 
\end{solution}
\end{questions}
\end{document}