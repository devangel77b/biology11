\documentclass[exam,addpoints,noanswers]{exam}

\usepackage[exam]{biology11}

\title{Practice Test \#4A}
\date{\today}
\author{\mobeardInstructorShort}

\begin{document}
\maketitle
\vfill
\mobeardExamNameBlock
\vfill
Instructions: 
\begin{enumerate}
\item Do not open exam until instructor announces that you may begin.
\item Closed notes, closed book.  You may use two \SI{8.5x11}{\inch} hardcopy note sheets, both sides.
\item Please write your name on each page in case the sheets are separated. 
\end{enumerate}
\vfill
\begin{center}
\gradetable[h][questions]
\end{center}
\clearpage

\begin{questions}
\question[1] What is mitosis? 
\begin{choices}
\choice The ten digits on the ends of my posterior appendages. 
\choice A process by which a diploid nucleus splits into four non-identical haploid nuclei.
\CorrectChoice A process by which a diploid nucleus splits into two identical diploid nuclei. 
\choice A disease caused by malnutrition originally described in engineering students in Cambridge, MA
\choice What's in a name? That which we call a rose by any other name would still smell as sweet. 
\end{choices}

\question[1] What are the phases of mitosis in the correct order?
\begin{choices}
\choice anaphase, metaphase, prophase, telophase
\choice prophase, metaphase, telophase, anaphase
\CorrectChoice prophase, metaphase, anaphase, telophase
\choice metaphase, prophase, anaphase, telophase
\end{choices}

\question[1] Which of the following are correct? Select all that apply.
\begin{choices}
\CorrectChoice During prophase, the chromosomes condense. 
\CorrectChoice During metaphase, the chromosomes line up along the equator.
\CorrectChoice During anaphase, the nuclear membrane dissolves.
\CorrectChoice During anaphase, the chromosomes are split and pulled along the spindle towards opposite poles.
\CorrectChoice During telophase, the nuclear membrane re-forms, resulting in two daughter nuclei.
\end{choices}

\question[1] What is meiosis? 
\begin{choices}
\choice An older, female sibling. 
\CorrectChoice A process by which a diploid nucleus splits into four non-identical haploid nuclei.
\choice A process by which a diploid nucleus splits into two identical haploid nuclei. 
\choice A disease caused by malnutrition from eating only mayonnaise. 
\choice What's in a name? That which we call a rose by any other name would still smell as sweet. 
\end{choices}

\question[1] Do all genes code for proteins? 
\begin{choices}
\choice Yes, otherwise they are not genes
\choice No, there is also ``junk DNA'' that does not code for anything.
\choice No, there are also regulatory genes
\choice Two of the above
\choice None of the above
\end{choices}

\clearpage
\question[1] Please explain two ways that meiosis generates genetic variation compared to asexual reproduction (e.g. by mitosis only). 
\begin{solution}[1in]
(1) Independent assortment, shuffling what alleles are present; and (2) recombination during crossover. 
\end{solution}

\question[1] Which of the following statements regarding evolution is \textbf{FALSE}?
\begin{choices}
\CorrectChoice Evolution has a ``drive'' and always acts to improve performance.
\choice Evolution is descent with modification.
\choice Evolution is a consequence of having heritable traits and differential survival of offspring.
\choice Evolution occurs over geologic timescales, but can also occur rapidly such as in COVID-19.
\end{choices}

\question[1] Based on \emph{scientific evidence}, approximately how old is life on earth?
\begin{choices}
\CorrectChoice Prokaryotic life is around 3.9 billion years old, multicellular organisms have only been around for about 500 million years.
\choice 5000 years.
\choice 17 years.
\choice Everything is 1 day old and was created yesterday by an evil genius out to trick us. 
\end{choices}

\question[1] Sickle cell anemia is a blood disorder resulting in reduced ability to carry oxygen in the blood. It is caused by having two copies of a recessive gene. If this condition is so bad, why might the gene persist in the gene pool, and how would you test this hypothesis?
\begin{solution}[1in]
\end{solution}

\question[1] (Short answer) Do genes determine 100\% how an organism will develop? Support your answer with examples.
\begin{solution}[2in]
\end{solution}



\clearpage
\question[23] If flies with straight wings (wild type) are crossed with flies with wrinkled wings, all the $f_1$ progeny have straight wings. Using a Punnett square, predict the phenotype of the $f_2$ progeny and their relative proportions.
\begin{solution}
Since no flies with wrinkled wings appear as a result of the $P_1$ cross, we will assume the parents were homozygous. Since all $f_1$ progeny had straight wings, we will also assume that straight wings ($S$) are dominant over wrinkled wings ($s$). The $f_1$ progeny, being heterozygous ($Ss$), will produce both $S$ and $s$ gametes in equal numbers. Therefore, the Punnett square for the f1 cross would be:
\begin{center}
\begin{tabular}{|c|c|c|}
\hline
& S & s \\
\hline
S & SS & Ss \\
\hline
s & Ss & ss \\
\hline
\end{tabular}
\end{center}
Since S is dominant over s, all genotypes containing even one S will have straight wings; only ss genotypes will express the wrinkled-wing phenotype. Thus, the f2 progeny will show a typical Mendelian dominant-recessive relationship of 3:1.
\end{solution}

\clearpage
\question[22] A mark of Mendel’s care and persistence is that he performed a test cross on each of the tall plants that arose from the $f_2$ of his original cross with tall and short plants. What ratio of tall to short plants would you expect him to have obtained from his test crosses?
\begin{solution}
Mendel’s $f_2$ generation had genotypes in the ratios of 1 $TT$:2 $Tt$:1 $tt$. Looking only at the tall plants produced in this generation (the only plants involved in the test cross), we can see that there are twice as many heterozygotes as homozygotes. Another way of viewing this fact is that, relative to the homozygotes, the heterozygotes produce twice as many gametes and therefore twice as many alleles as the tall homozygotes.

Now, consider how many $T$ alleles versus $t$ alleles there are in the tall population. The homozygous tall plant contributes two $T$ alleles, so there are 2x$T$ alleles contributed by the homozygotes, where x is the total number of homozygous plants. The heterozygous tall plant contributes only one $T$ allele; however, since there are twice as many heterozygotes, these plants also contribute 2x $T$ alleles. The heterozygote also produces a $t$ gamete, and the whole population of heterozygotes produces 2x$t$ alleles. Comparing $T$ and $t$ alleles, there are 4x$T$ alleles (2x + 2x) and 2x $t$ alleles in the population, or a ratio of 2:1. Each of these alleles will combine with a t allele from the homozygous short plant in the test cross; a $T$ allele from the population being tested will produce a tall plant ($Tt$), and a $t$ allele will produce a short plant (tt). Since the ratio of $T$ alleles to t alleles is 2:1, this will also be the ratio of tall to short plants we would expect from the test crosses. This is, in fact, what Mendel obtained and provides clear verification of the law of segregation -- a law with its roots in the operation of chance for the distribution of the discrete particulate alleles.
\end{solution}

\clearpage
\question[23] Sketch what you expect to see when observing mitosis. Make four sketches, one for each of the phases we discussed. For each, label important structures and add a caption explaining the action happening. 

\clearpage
\question[22] Humans live lives in which the gametes are haploid, they fuse to form a diploid zygote which develops into an adult human.  Is this lifecycle the only way to live? If not please give examples and explain. 


\end{questions}
\end{document}