\documentclass [handout]{tufte-handout}

\usepackage{biology11}

\title{Q4 study guide}
\author{\mobeardInstructorShort}
\date{\today}

\begin{document}
\maketitle

The Q4 test will be on Tuesday, \printdate{6/2/2021}. You are allowed two (2) sheets of \SI{8.5x11}{\inch} hardcopy notes (both sides). Please do not plan to use your computer or phone for your notes. I think I plan to do this exam on paper for those who are present; on GoFormative only for virtual students. 
\marginnote{The primary focus of the Q4 test is Q4 but I reserve the right to ask anything from Q3 that I think you didn't do well enough on the first time, to check that you have picked it up since then.}

\section{Mitosis}
Know what \textbf{mitosis} does and how it fits in the \textbf{cell cycle}. You should feel comfortable being asked to draw and explain the four stages; or, if shown drawings of the four stages, to identify them and explain what is going on. \marginnote{Mr.~Perka in Manalapan snapping a meter stick and shouting prophase metaphase anaphase telophase!}

\section{Meiosis}
Know what \textbf{meiosis} does. You should feel comfortable being asked to draw and explain the stages; or, if shown drawings of the stages, to identify them and explain what is going on. You should know in general terms how meiosis fits in the life cycle of sexually reproducing species\sidenote{Evangelista always wants to know how $x$ fits into larger biology questions.}. Potential bonus questions could include why having extra or missing chromosomes can mess up meiosis, why some hybrid animals are sterile, etc. \marginnote{Note similarities with mitosis to help streamline your studying}

Be able to explain the terms \textbf{haploid} and \textbf{diploid}. Potential bonus questions could include explaining a life cycle that is different from humans. 

\section{Really basic genetics}
Know how to do $F_1$ and $F_2$ crosses for a \textbf{plain vanilla cross like Mendel's green and yellow peas}. Know the terms $P$, $F_1$, $F_2$, homozygous, heterozygous, allele, genotype, phenotype... Be able to draw a \textbf{Punnet square} and use it to obtain the \textbf{predicted ratios of genotypes versus the predicted ratios of phenotypes} for simple vanilla crosses. 

Potential bonus questions include the ideas beyond\sidenote{Evangelista always wants to know about exceptions or differences from the simple version} plain vanilla genetics, as discussed in the in class exercise we did (sex linked examples, lethal examples); epistasis (e.g. Labrador coat color); polygenic inheritance; multiple dominants; etc. \textbf{Hardy-Weinberg} will not be covered. 
\end{document}